\documentclass[11pt,a4paper]{article}
\usepackage[spanish,es-nodecimaldot]{babel}	% Utilizar español
\usepackage[utf8]{inputenc}					% Caracteres UTF-8
\usepackage{graphicx}						% Imagenes
\usepackage[hidelinks]{hyperref}			% Poner enlaces sin marcarlos en rojo
\usepackage{fancyhdr}						% Modificar encabezados y pies de pagina
\usepackage{float}							% Insertar figuras
\usepackage[textwidth=390pt]{geometry}		% Anchura de la pagina
\usepackage[nottoc]{tocbibind}				% Referencias (no incluir num pagina indice en Indice)
\usepackage{enumitem}						% Permitir enumerate con distintos simbolos
\usepackage[T1]{fontenc}					% Usar textsc en sections
\usepackage{amsmath}						% Símbolos matemáticos
\usepackage{amssymb}
\usepackage{natbib}
\usepackage{xcolor}
\usepackage{algorithm}
\usepackage[noend]{algpseudocode}			% Pseudocodig

\algrenewcommand\algorithmicreturn{\textbf{return}}

% Comando para poner el nombre de la asignatura
\newcommand{\asignatura}{Visión por Computador}
\newcommand{\autor}{Vladislav Nikolov Vasilev}
\newcommand{\titulo}{Trabajo 3}
\newcommand{\subtitulo}{Cuestiones de teoría}

\newcommand{\answer}{\noindent\textbf{Solución}}
\newcommand{\question}[1]{\noindent\textbf{#1}}
\newcommand{\nonumsection}[1]{\section*{#1}\addcontentsline{toc}{section}{#1}}

% Configuracion de encabezados y pies de pagina
\pagestyle{fancy}
\lhead{\autor{}}
\rhead{\asignatura{}}
\lfoot{Grado en Ingeniería Informática}
\cfoot{}
\rfoot{\thepage}
\renewcommand{\headrulewidth}{0.4pt}		% Linea cabeza de pagina
\renewcommand{\footrulewidth}{0.4pt}		% Linea pie de pagina

\begin{document}
\pagenumbering{gobble}

% Pagina de titulo
\begin{titlepage}

\begin{minipage}{\textwidth}

\centering

\includegraphics[scale=0.5]{img/ugr.png}\\

\textsc{\Large \asignatura{}\\[0.2cm]}
\textsc{GRADO EN INGENIERÍA INFORMÁTICA}\\[1cm]

\noindent\rule[-1ex]{\textwidth}{1pt}\\[1.5ex]
\textsc{{\Huge \titulo\\[0.5ex]}}
\textsc{{\Large \subtitulo\\}}
\noindent\rule[-1ex]{\textwidth}{2pt}\\[3.5ex]

\end{minipage}

\vspace{0.5cm}

\begin{minipage}{\textwidth}

\centering

\textbf{Autor}\\ {\autor{}}\\[2.5ex]
\textbf{Rama}\\ {Computación y Sistemas Inteligentes}\\[2.5ex]
\vspace{0.3cm}

\includegraphics[scale=0.3]{img/etsiit.jpeg}

\vspace{0.7cm}
\textsc{Escuela Técnica Superior de Ingenierías Informática y de Telecomunicación}\\
\vspace{1cm}
\textsc{Curso 2019-2020}
\end{minipage}
\end{titlepage}

\pagenumbering{arabic}
\tableofcontents
\thispagestyle{empty}				% No usar estilo en la pagina de indice

\newpage

\setlength{\parskip}{1em}

\nonumsection{Ejercicio 1}

\question{¿Cuál es la transformación más fuerte de la geometría de una
escena que puede introducirse al tomar una foto de ella? Dar algún
ejemplo.}

\answer

La transformación más fuerte de la geometría de una escena que puede introducirse
al tomar una foto es la \textbf{transformación proyectiva}, también conocida como
\textbf{transformación de perspectiva}. Esta transformación es capaz de conservar
las líneas rectas, pero no conserva ni los ángulos, ni los tamaños de los objetos
de la escena ni las líneas paralelas, ya que estos elementos dependen de la perspectiva
del observador dentro de la escena. Dos observadores situados en la misma escena
pero en distintos lugares no tomarán la misma foto si observan al mismo punto.

Un ejemplo muy claro de esto es la típica foto de las vías del tren. En la realidad,
las vías forman líneas rectas, pero al tomar una fotografía de ellas desde el suelo,
las líneas no son paralelas; es más, parece que se cortan en el infinito (cosa que
en la realidad no sucede). Otro ejemplo sería por ejemplo una foto tomada en un pasillo.
Las líneas que separan el suelo de las paredes de cada lado del pasillo no se juntan
en la realidad en ningún momento, pero por la perspectiva con la que se ha tomado la foto,
tal y como pasó en el caso de las vías, podría parecer que las líneas se juntasen en el
infinito.

\nonumsection{Ejercicio 2}

\question{¿Por qué es necesario usar el plano proyectivo para estudiar las
transformaciones en las imágenes de fotos de escenas? Dar algún
ejemplo.}

\answer

Necesitamos utilizar el plano proyectivo para estudiar las transformaciones
en las imágenes de fotos de escenas porque existen ciertas transformaciones, como
por ejemplo las cambios de perspectiva, los cuáles se dan al tomar fotos de
una misma escena desde distintas posiciones, que no se pueden estudiar en el
plano afín, ya que son un tipo de transfomración que rompe con ciertas propiedades
geométricas que tienen los planos afines, como por ejemplo la conservación
del paralelismo.

Por tanto, para estudiar dichas transformaciones necesitamos hacerlo en el
plano proyectivo, ya que ahí las transformaciones se pueden representar
como \textbf{homografías}. Ĺas homografías, no obstante, no solo representan transformaciones
de perspectiva, sino que también se pueden utilizar para representar transformaciones
afines. Por tanto, realmente se puede estudiar cualquier tipo de transformación en
el plano proyectivo, lo cuál lo hace más adecuado y preferente sobre el afín.

\nonumsection{Ejercicio 3}

\question{Sabemos que en el plano proyectivo un punto no existe en el
sentido del plano afín, sino que se define por una clase de
equivalencia de vectores definida por $\lbrace k(x,y,1),k \neq 0 \rbrace$. Razone usando
las coordenadas proyectivas de los puntos afines de una recta que
pase por el $(0,0)$ del plano afín y verifique que los punto de la
recta del infinito del plano proyectivo son necsariamente vectores
del tipo $(*,*,0)$ con $*=$cualquier número.}

\answer

\nonumsection{Ejercicio 4}

\question{¿Qué propiedades de la geometría de un plano quedan invariantes
cuando se toma una foto de él? Justificar la respuesta.}

\answer

Cuando tomamos una fotografía de un plano, se conservan las siguientes
propiedades:

\begin{itemize}[label=\textbullet]
	\item \textbf{Líneas rectas}: Las líneas rectas se conservan, ya que
	tomar una foto de un plano desde un punto u otro no produce una deformación
	de estas.
	\item \textbf{Colinealidad}: Los puntos que pertenecen a alguna línea seguirán
	perteneciendo a ella, y aquellos que no, seguirán sin pertenecer, independientemente
	del punto de vista en el que se sitúe el observador.
\end{itemize}

\nonumsection{Ejercicio 5}

\question{En coordenadas homogéneas los puntos y rectas del plano se representan por vectores de tres coordenadas (notados x y l respectivamente), de manera que si una recta contiene a un punto se verifica la ecuación $x^Tl=0$, es decir $(x_1, x_2, x_3) \left( \begin{matrix} a \\ b \\ c \end{matrix} \right) = 0$. Considere una homografía H que transforma vectores de puntos, $x^\prime = Hx$. Dado que una homografía transforma vectores de tres coordenadas también existen homografías G para transformar vectores de rectas $l^\prime = Gl$. Suponga una recta l y un punto x que verifican $x^Tl=0$ en el plano proyectivo y suponga que conoce una homografía H que transforma vectores de puntos. En estas condiciones ¿cuál es la homografía G que transforma los vectores de las rectas? Deducirla matemáticamente.}

\answer

\nonumsection{Ejercicio 6}

\question{¿Cuál es el mínimo número de escalares necesarios para fijar
una homografía general? ¿Y si la homografía es afín? Justificar la
respuesta.}

\answer

\nonumsection{Ejercicio 7}

\question{Defina una homografía entre planos proyectivos que haga que el
punto $(3,0,2)$ del plano proyectivo-1 se transforme en un punto de
la recta del infinito del plano proyectivo-2? Justificar la
respuesta.}

\answer

Partiendo de que los puntos del plano proyectivo-2 son de la forma
$(x', y', 0)$, tenemos que encontrar una homografía que nos permita
llevar el punto $(3,0,2)$ a una recta de dicho plano, haciendo que el
último valor sea por tanto un 0 en vez de un 2. Para ello, vamos a plantear esta
transformación de la siguiente forma:

\begin{equation}
	\begin{pmatrix}
 	a & b & c \\ 
 	d & e & f \\ 
 	g & h & i 
	\end{pmatrix}
	\begin{pmatrix}
 	3 \\ 
 	0 \\ 
 	2
	\end{pmatrix}
	=
	\begin{pmatrix}
 	x' \\ 
 	y' \\ 
 	0
	\end{pmatrix}
\end{equation}

\noindent donde la homografía $\mathbf{H}$ es:

\begin{equation}
\mathbf{H} = 	\begin{pmatrix}
 	a & b & c \\ 
 	d & e & f \\ 
 	g & h & i 
	\end{pmatrix}
\end{equation}

Realmente de aquí solo nos interesa calcular los valores de $g$, $h$ e $i$
que nos permiten transformar el 2 que se tiene en un principio a 0.
Para ello, vamos a reescribir la última parte como una ecuación.
La expresión es la siguiente:

\begin{equation}
	3g + 0h + 2i = 0
\end{equation}

Simplificando la expresión anterior, obtenemos lo siguiente:

\begin{equation}
	3g + 2i = 0
\end{equation}

La expresión anterior tiene realmente infinitas soluciones,
aunque todas ellas se pueden expresar de la siguietne forma:

\[
	g = 2k \; , \; i = -3k \;, \; donde \; k \in \mathbb{Z}
\]

Por tanto, podríamos escoger como una solución simple $g = 2$
e $i = -3$. El valor de $h$ sería $h = 0$, y el resto de elementos
de $\mathbf{H}$ podríamos ponerlos como la matriz identidad. Por
tanto, tendríamos lo siguiente:

\begin{equation}
\mathbf{H} = \begin{pmatrix}
 	1 & 0 & 0 \\ 
 	0 & 1 & 0 \\ 
 	2 & 0 & -3 
	\end{pmatrix}
\end{equation}

Ahora, si queremos calcular las coordenadas del nuevo punto, simplemente
tenemos que palicar la homografía calculada anteriormente, y obtendríamos
que:

\begin{equation}
	\begin{pmatrix}
 	1 & 0 & 0 \\ 
 	0 & 1 & 0 \\ 
 	2 & 0 & -3 
	\end{pmatrix}
	\begin{pmatrix}
 	3 \\ 
 	0 \\ 
 	2
	\end{pmatrix}
	=
	\begin{pmatrix}
 	3 \\ 
 	0 \\ 
 	0
	\end{pmatrix}
\end{equation}


\nonumsection{Ejercicio 8}

\question{Una homografía general $\mathbf{H}=\begin{pmatrix}a & b & c \\ d & e & f \\
g & h & i \end{pmatrix} = \begin{bmatrix} \mathbf{A} & \mathbf{t} \\ \mathbf{v}^T & v
\end{bmatrix}$, $\text{det}(\mathbf{H}) \neq 0$ admite una
descomposición única en movimiento elementales de la siguiente
forma $\mathbf{H} = \mathbf{H}_S\mathbf{H}_A\mathbf{H}_P$ donde $\mathbf{H}_S$
representa la homografía de una similaridad
(escala, giro y traslación), $\mathbf{H}_A$ la homografía de un movimiento afín
puro y $\mathbf{H}_P$ una transformación proyectiva pura. Es decir,}

$$
\mathbf{H}_S = \begin{pmatrix} s\cos\theta & -s\sin\theta & t_x \\ s\sin\theta & s\cos\theta & t_y \\
0 & 0 & 1 \end{pmatrix} \equiv \begin{bmatrix} s\mathbf{R} & \mathbf{t} \\ \mathbf{0}^T & 1 \end{bmatrix}, s>0
$$

$$
\mathbf{H}_A = \begin{pmatrix} a & c & 0 \\ 0 & b &0 \\ 0 & 0 & 1 \end{pmatrix} \equiv
\begin{bmatrix} s\mathbf{K} & \mathbf{0} \\ \mathbf{0}^T & 1 \end{bmatrix}, \text{det}(\mathbf{K}=1
$$

$$
\mathbf{H}_P = \begin{pmatrix} 1 & 0 & 0 \\ 0 & 1 &0 \\ v_1 & v_2 & v \end{pmatrix} \equiv
\begin{bmatrix} \mathbf{I} & \mathbf{0} \\ \mathbf{v}^T & v \end{bmatrix}, v \neq 0
$$

\question{(Notación: en negrita son vectores o matrices)}

\question{Describir un algoritmo que permite encontrar las matrices de la
descomposición de una matriz $\mathbf{H}$ dada. Aplicarlo para encontrar la
descomposición de}

$$
	\mathbf{H} = \begin{pmatrix}
	1.707 & 0.586 & 1.0 \\
	2.707 & 8.242 & 2.0 \\
	1.0 & 2.0 & 1.0
	\end{pmatrix}
$$

\answer

Para empezar, voy a decir que he consultado el libro
\textit{Multiple View Geometry in Computer Vision}\cite{Hartley:2003:MVG:861369}
para extraer información y para guiarme, ya que el problema
planteado puede ser encontrado ahí. En la solución primero
mostraremos cuáles son las bases de lo que vamos a hacer, y después
mostraremos un pseudocódigo junto con el resultado de la descomposición
propuesta.

Vamos a empezar sacando la información más inmediata. Si miramos la expresión
$\mathbf{H}=\begin{pmatrix}a & b & c \\ d & e & f \\
g & h & i \end{pmatrix} = \begin{bmatrix} \mathbf{A} & \mathbf{t} \\ \mathbf{v}^T & v
\end{bmatrix}$ podemos deducir una serie de cosas:

\begin{enumerate}
	\item $\mathbf{A}$ tiene que ser una matriz $2\times2$, y tiene que
	equivaler a $\begin{pmatrix}a & b \\ d & e \end{pmatrix}$.
	\item $\mathbf{t}$ es, por la notación, un vector columna. Por tanto,
	tiene que tener el mismo número de filas que $\mathbf{A}$, y por tanto,
	tiene que tener un tamaño de $2 \times 1$. De esta forma, tendríamos
	que $\mathbf{t} = \begin{pmatrix} c \\ f \end{pmatrix}$.
	\item $\mathbf{v}^T$ es, por la notación, un vector fila. Por tanto,
	tiene que tener el mismo número de columnas que $\mathbf{A}$, y por tanto,
	tiene que tener un tamaño de $1 \times 2$. De esta forma, tendríamos
	que $\mathbf{v}^T = \begin{pmatrix} g & h \end{pmatrix}$.
	\item $v$ es un único valor, ya que no está expresado en notación de vector
	y/o matriz. Por tanto, el único caso que puede darse es $v = i$.
\end{enumerate}

Por tanto, de aquí ya podemos extraer algunos elementos de la descomposición,
los cuáles son $\mathbf{t}$, $\mathbf{v}^T$ y $v$. $\mathbf{A}$, en un principio, también
lo tenemos, pero si nos fijamos en las matrices que forman parte de la descomposición,
vemos que no aparece por ninguna parte. En su lugar, aparecen otras expresiones, como
$s\mathbf{R}$ y $\mathbf{K}$. Por tanto, tenemos que encontrar alguna manera
de calcular estas dos matrices, de manera que, con las expresiones que obtengamos,
podamos crear un algoritmo para hacer la descomposición.

Si miramos el libro, podemos encontrar
una expresión para obtener $\mathbf{A}$, la cuál es la siguiente:

\begin{equation}
\label{dec}
	\mathbf{A} = s\mathbf{RK} + \mathbf{tv}^T
\end{equation}

\noindent donde $\mathbf{K}$ es una matriz triangular superior con $\text{det}(\mathbf{K})=1$.

Esta expresión surge de ir multiplicando las expresiones que aparecen
entre corchetes. Para demostrarlo, vamos a ver dicho desarrollo:

\begin{gather*}
\begin{bmatrix} \mathbf{A} & \mathbf{t} \\ \mathbf{v}^T & v
\end{bmatrix} =
\begin{bmatrix} s\mathbf{R} & \mathbf{t} \\ \mathbf{0}^T & 1 \end{bmatrix}
\begin{bmatrix} s\mathbf{K} & \mathbf{0} \\ \mathbf{0}^T & 1 \end{bmatrix}
\begin{bmatrix} \mathbf{I} & \mathbf{0} \\ \mathbf{v}^T & v \end{bmatrix}
=
\begin{bmatrix} s\mathbf{RK} & \mathbf{t} \\ \mathbf{0}^T & 1 \end{bmatrix}
\begin{bmatrix} \mathbf{I} & \mathbf{0} \\ \mathbf{v}^T & v \end{bmatrix}
\\ =
\begin{bmatrix} s\mathbf{RK} + \mathbf{tv}^T & \mathbf{t} \\ \mathbf{v}^T & v \end{bmatrix}
\end{gather*}

Como podemos ver, la expresión anterior aparece dentro de la matriz
resultante.

Nuestro objetivo es determinar las expresiones para calcular
$s\mathbf{R}$ y $\mathbf{K}$. Para obtener dichos valores, nos
vamos a ayudar de aquellos que ya conocemos, que son
$\mathbf{t}$, $\mathbf{v}^T$, $v$ y $\mathbf{A}$.

Para empezar, vamos a reescribir la expresión que se ve en \eqref{dec}
con el objetivo de simplificar cálculos futuros:

\begin{equation}
	 s\mathbf{RK} = \mathbf{A} - \mathbf{tv}^T
\end{equation}

Ahora, si desarrollamos la expresión anterior con toda la información
que conocemos, obtenemos lo siguiente:

\begin{gather*}
s\mathbf{RK} = \mathbf{A} - \mathbf{tv}^T \\
\begin{pmatrix}
s\cos\theta & -s\sin\theta \\
s\sin\theta & s\cos\theta
\end{pmatrix}
\begin{pmatrix}
k_a & k_c \\
0 & k_b
\end{pmatrix}
=
\begin{pmatrix}
a & b \\ d & e
\end{pmatrix}
-
\begin{pmatrix}
c \\ f
\end{pmatrix}
\begin{pmatrix}
g & h
\end{pmatrix}
\\
\begin{pmatrix}
k_as\cos\theta & k_cs\cos\theta - k_bs\sin\theta \\
k_as\sin\theta & k_cs\sin\theta + k_bs\cos\theta
\end{pmatrix}
=
\begin{pmatrix}
a & b \\ d & e
\end{pmatrix}
-
\begin{pmatrix}
cg & ch \\
fg & fh
\end{pmatrix}
\\
\begin{pmatrix}
k_as\cos\theta & k_cs\cos\theta - k_bs\sin\theta \\
k_as\sin\theta & k_cs\sin\theta + k_bs\cos\theta
\end{pmatrix}
=
\begin{pmatrix}
a-cg & b-ch \\ d-fg & e-fh
\end{pmatrix}
\end{gather*}

Podemos plantear dicha expresión como un sistema de ecuaciones:

\begin{equation}
\left.\begin{matrix}
 k_as\cos\theta = a-cg \\
 k_cs\cos\theta - k_bs\sin\theta = b -ch \\
 k_as\sin\theta = d-fg \\
 k_cs\sin\theta + k_bs\cos\theta = e-fh
\end{matrix}\right\}
\end{equation}

De aquí, nuestro objetivo es obtener los valores de $s$, $\theta$,
$k_a$, $k_b$ y $k_c$. Si nos fijamos, de momento tenemos un sistema
con cuatro ecuaciones y cinco incógnitas. En un principio, tendríamos
muchísimos problemas para resolverlo, ya que nos faltaría una ecuación
más. Afortunadamente, tiene que cumplirse que el determinante de $\mathbf{K}$ tiene que
ser 1. Para que esto suceda, tiene que darse que $k_a k_b = 1$. Por tanto, tiene que cumplirse
que

$$\textcolor{blue}{k_b = \frac{1}{k_a}}$$

Sustituyendo, obtenemos lo siguiente:

\begin{equation}
\left.\begin{matrix}
 k_as\cos\theta = a-cg \\
 k_cs\cos\theta - \frac{1}{k_a}s\sin\theta = b -ch \\
 k_as\sin\theta = d-fg \\
 k_cs\sin\theta + \frac{1}{k_a}s\cos\theta = e-fh
\end{matrix}\right\}
\end{equation}

Ahora tenemos un sistema de cuatro ecuaciones con cuatro
incógnitas. Sin embargo, obtener las soluciones nos va a llevar
bastante trabajo, ya que queremos que todas ellas sean genéricas
para cualquier homografía $\mathbf{H}$, de forma que de aquí
podamos sacar algún algoritmo que nos pueda servir para
cualquier matriz que cumpla con las restricciones impuestas.

Lo primero que podemos intentar hacer es obtener el valor de $\theta$,
ya que parece ser lo más ``fácil'' de sacar. Para hacerlo, vamos
a utilizar la primera y la tercer ecuación, ya que ambas tienen
$k_a$ en ellas.

Si dejamos la primera ecuación en función de $k_a$, obtenemos
lo siguiente:

\begin{equation}
\label{simp1}
k_a = \frac{a-cg}{s\cos\theta}
\end{equation}

Si dejamos la tercera ecuación en función de $k_a$, obtenemos
la siguiente expresión:

\begin{equation}
\label{simp2}
k_a = \frac{d-fg}{s\sin\theta}
\end{equation}

Si igualamos, obtenemos que:

\begin{gather*}
\frac{a-cg}{s\cos\theta} = \frac{d-fg}{s\sin\theta} \\
s\sin\theta(a-cg) = s\cos\theta(d-fg) \\
\frac{s\sin\theta}{s\cos\theta} = \frac{d-fg}{a-cg} \\
\frac{\sin\theta}{\cos\theta} = \frac{d-fg}{a-cg} \\
\tan \theta = \frac{d-fg}{a-cg}
\end{gather*}

Ya que tenemos el valor de la tangente de $\theta$, para
obtener el valor de $\theta$ solo tenemos que aplicar
la función inversa, la cuál es el arco tangente. Por tanto,
tenemos que $\theta$ tiene el siguiente valor:

\begin{equation}
\textcolor{blue}{\theta = \arctan \Bigg( \frac{d-fg}{a-cg} \Bigg)}
\end{equation}

Ahora que ya tenemos $\theta$, aun nos resta encontrar los
valores de $k_a$, $k_c$ y $s$. Lo primero que podemos hacer
es dejar el valor de $s$ en función de $k_a$. Para hacer
esto, podemos aislar en la primera ecuación, y obtendríamos
la siguiente expresión:

\begin{equation}
\textcolor{blue}{s = \frac{a-cg}{k_a\cos\theta}}
\end{equation}

Ahora, vamos a intentar obtener el valor de $k_a$. Para
ello podemos intentar utilizar la segunda y la cuarta
ecuación, aunque para eso, vamos a simplificarlas antes:

\begin{equation}
\left.\begin{matrix}
 k_as\cos\theta = a-cg \\
 s\Big(k_c\cos\theta - \frac{1}{k_a}\sin\theta\Big) = b -ch \\
 k_as\sin\theta = d-fg \\
 s\Big(k_c\sin\theta + \frac{1}{k_a}\cos\theta\Big) = e-fh
\end{matrix}\right\}
\end{equation}

Ahora podemos coger la segunda y la cuarta ecuación,
teniendo por tanto el siguiente par de ecuaciones:

\begin{equation}
\left.\begin{matrix}
 s\Big(k_c\cos\theta - \frac{1}{k_a}\sin\theta\Big) = b -ch \\
 s\Big(k_c\sin\theta + \frac{1}{k_a}\cos\theta\Big) = e-fh
\end{matrix}\right\}
\end{equation}

Si aislamos $s$ en cada lado, obtenemos:

\begin{equation}
\left.\begin{matrix}
 s = \frac{b-ch}{k_c\cos\theta - \frac{1}{k_a}\sin\theta} \\
 s = \frac{e-fh}{k_c\sin\theta + \frac{1}{k_a}\cos\theta}
\end{matrix}\right\}
\end{equation}

Antes de continuar, para facilitar el trabajo,
vamos a hacer la siguiente sustitución de variables:

$$
t_1 = b-ch, t_2 = e-fh
$$

Ahora, si sustituimos por las variables anteriores e igualamos
las dos expresiones, obtenemos lo siguiente:

\begin{gather*}
\frac{t_1}{k_c\cos\theta - \frac{1}{k_a}\sin\theta} = \frac{t_2}{k_c\sin\theta + \frac{1}{k_a}\cos\theta}\\
t_1k_c\sin\theta + \frac{t_1\cos\theta}{k_a} = t_2k_c\cos\theta - \frac{t_2\sin\theta}{k_a}\\
t_1k_ak_c\sin\theta + t_1\cos\theta = t_2k_ak_c\cos\theta - t_2\sin\theta\\
t_2k_ak_c\cos\theta - t_1k_ak_c\sin\theta = t_1\cos\theta + t_2\sin\theta \\
k_ak_c = \frac{t_1\cos\theta + t_2\sin\theta}{t_2\cos\theta - t_1\sin\theta}\\
k_c = \frac{t_1\cos\theta + t_2\sin\theta}{t_2\cos\theta - t_1\sin\theta} \cdot \frac{1}{k_a}\\
\textcolor{blue}{k_c = \frac{(b-ch)\cos\theta + (e-fh)\sin\theta}{(e-fh)\cos\theta - (b-ch)\sin\theta} \cdot \frac{1}{k_a}}
\end{gather*}

Ahora si sustituimos $s$ y $k_c$ en la segunda ecuación por las
expresiones que hemos obtenido hasta ahora, tenemos lo siguiente:

\begin{gather*}
\frac{a-cg}{k_a\cos\theta}\Bigg(\frac{t_1\cos\theta + t_2\sin\theta}{t_2\cos\theta - t_1\sin\theta} \cdot \frac{\cos\theta}{k_a} - \frac{\sin\theta}{k_a}\Bigg) = t_1\\
\frac{a-cg}{k_a^2\cos\theta}\Bigg(\frac{t_1\cos\theta + t_2\sin\theta}{t_2\cos\theta - t_1\sin\theta} \cdot \cos\theta - \sin\theta\Bigg) = t_1\\
\frac{a-cg}{k_a^2}\Bigg(\frac{t_1\cos\theta + t_2\sin\theta}{t_2\cos\theta - t_1\sin\theta} - \tan\theta\Bigg) = t_1\\
k_a^2=\frac{a-cg}{t_1}\Bigg(\frac{t_1\cos\theta + t_2\sin\theta}{t_2\cos\theta - t_1\sin\theta} - \tan\theta\Bigg)\\
k_a=\sqrt{\frac{a-cg}{t_1}\Bigg(\frac{t_1\cos\theta + t_2\sin\theta}{t_2\cos\theta - t_1\sin\theta} - \tan\theta\Bigg)}\\
\textcolor{blue}{k_a=\sqrt{\frac{a-cg}{b-ch}\Bigg(\frac{(b-ch)\cos\theta + (e-fh)\sin\theta}{(e-fh)\cos\theta - (b-ch)\sin\theta} - \tan\theta\Bigg)}}
\end{gather*}

Y con esto, ya tendríamos las expresiones necesarias para calcular
todas las incógnitas. Si nos fijamos en las expresiones anteriores, vemos
que algunas de ellas están en azul. Esto se debe a que estas serán las expresiones
que utilizaremos en nuestro algoritmo para calcular la descomposición.

Por tanto, sin más dilación, vamos a ver un pseudocódigo del algoritmo para
calcular la descomposición de la matriz $\mathbf{H}$:

\begin{algorithm}[H]
\caption{Pseudocódigo de la descomposición de $\mathbf{H}$}
\begin{algorithmic}[1]
\Function{DescomponerH}{$\mathbf{H}$}
\State $a,b,c,d,e,f,g,h,i \gets \mathbf{H}_{11}, \mathbf{H}_{12}, \mathbf{H}_{13}, \mathbf{H}_{21}, \mathbf{H}_{22}, \mathbf{H}_{23}, \mathbf{H}_{31}, \mathbf{H}_{32}, \mathbf{H}_{33}$
\State $\theta \gets \arctan\Big( \frac{d-fg}{a-cg} \Big)$
\State $k_a \gets \sqrt{\frac{a-cg}{b-ch}\Big(\frac{(b-ch)\cos\theta + (e-fh)\sin\theta}{(e-fh)\cos\theta - (b-ch)\sin\theta} - \tan\theta\Big)}$
\State $k_b \gets \frac{1}{k_a}$
\State $k_c \gets \frac{(b-ch)\cos\theta + (e-fh)\sin\theta}{(e-fh)\cos\theta - (b-ch)\sin\theta} \cdot \frac{1}{k_a}$
\State $s \gets \frac{a-cg}{k_a\cos\theta}$
\State $\mathbf{H}_S \gets \begin{pmatrix} s\cos\theta & -s\sin\theta & c \\ s\sin\theta & s\cos\theta & f \\
0 & 0 & 1 \end{pmatrix}$
\State $ \mathbf{H}_A \gets \begin{pmatrix}
k_a & k_c & 0 \\ 0 & k_b & 0 \\ 0 & 0 & 1
\end{pmatrix}
$
\State $ \mathbf{H}_P \gets
\begin{pmatrix}
1 & 0 & 0 \\ 0 & 1 & 0 \\ g & h & i
\end{pmatrix}
$
\State \Return $\mathbf{H}_S$, $\mathbf{H}_A$, $\mathbf{H}_P$
\EndFunction
\end{algorithmic}
\end{algorithm}

Vamos a aplicar ahora el algoritmo para descomponer la matriz
siguiente:

$$
	\mathbf{H} = \begin{pmatrix}
	1.707 & 0.586 & 1.0 \\
	2.707 & 8.242 & 2.0 \\
	1.0 & 2.0 & 1.0
	\end{pmatrix}
$$

\begin{enumerate}
	\item Obtenemos los valores iniciales:
	$$
	a = 1.707, b=0.586, c = 1.0, d=2.707, e = 8.242, f = 2.0, g=1.0, h=2.0, i=1.0
	$$
	\item Calculamos $\theta$:
	
	$$
	\theta = \arctan\Big( \frac{2.707-2.0}{1.707-1.0} \Big) =
	\arctan\Big( \frac{0.707}{0.707} \Big) = \arctan(1) = \frac{\pi}{4}
	$$
	
	\item Calculamos $k_a$:
	
	\begin{gather*}
	k_a = \sqrt{\frac{1.707-1}{0.586-2}\Bigg(\frac{(0.586-2)\cos\frac{\pi}{4} + (8.242-4)\sin\frac{\pi}{4}}{(8.242-4)\cos\frac{\pi}{4} - (0.586-2)\sin\frac{\pi}{4}} - \tan\frac{\pi}{4}\Bigg)} = 0.5
	\end{gather*}
	
	\item Calculamos $k_b$:
	
	$$
	k_b = \frac{1}{0.5} = 2 
	$$
	
	\item Calculamos $k_c$:
	
	$$
	k_c = \frac{(0.586-2)\cos\frac{\pi}{4} + (8.242-4)\sin\frac{\pi}{4}}{(8.242-4)\cos\frac{\pi}{4} - (0.586-2)\sin\frac{\pi}{4}}  \cdot \frac{1}{0.5} = 1
	$$
	
	\item Calculamos $s$:
	
	$$
	s = \frac{1.707 - 1}{0.5\cdot \cos\frac{\pi}{4}} = 1.99969 \approx 2
	$$
	
	\item Obtenemos las matrices:
	
	\begin{gather*}
	\mathbf{H}_S =
	\begin{pmatrix}
	2\cos\frac{\pi}{4} & -2\sin\frac{\pi}{4} & 1\\
	2\sin\frac{\pi}{4} & 2\cos\frac{\pi}{4} & 2\\
	0 & 0 & 1
	\end{pmatrix}\\
	\mathbf{H}_A =
	\begin{pmatrix}
	0.5 & 1 & 0 \\ 0 & 2 & 0 \\ 0 & 0 & 1
	\end{pmatrix}\\
	\mathbf{H}_P =
	\begin{pmatrix}
	1 & 0 & 0 \\ 0 & 1 & 0 \\ 1 & 2 & 1
	\end{pmatrix}
 	\end{gather*}
\end{enumerate}

\nonumsection{Ejercicio 9}

\question{¿Cuáles son las propiedades necesarias y suficientes para que
una matriz defina un movimiento geométrico no degenerado entre
planos? Justificar la respuesta.}

\answer

\nonumsection{Ejercicio 10}

\question{¿Qué información de la imagen usa el detector de Harris para
seleccionar puntos? ¿El detector de Harris detecta patrones
geométricos o fotométricos? Justificar la contestación.}

\answer

Harris utiliza información sobre el gradiente para seleccionar puntos.
Para cada píxel, se obtienen los valores singulares de una matriz $2\times2$,
los cuáles son $\lambda_1$ y $\lambda_2$ y están ligados a los valores
de los gradientes en el eje $X$ y en el eje $Y$, respectivamente.
Para cada píxel se obtiene un valor $f = \frac{\lambda_1 \lambda_2}{\lambda_1 + \lambda_2}$,
el cuál representa el valor del punto Harris. Según los
valores de $\lambda_1$ y $\lambda_2$ se determina a qué tipo de región
pertenece el píxel:

\begin{itemize}[label=\textbullet]
	\item Si los dos valores son pequeños, nos encontramos en una región plana.
	\item Si un valor es grande y el otro pequeño, significa que estamos en
	un borde, ya que la intensidad varía mucho solo en uno de los ejes.
	\item Si los dos valores son grandes, significa que estamos en una esquina,
	ya que la intensidad varía mucho en ambos ejes.
\end{itemize}

Harris detecta principalmente patrones geométricos, ya que detecta esquinas
de objetos a partir de la información de los gradientes. Sin embargo,
al utilizar información del gradiente, está obteniendo información procedente
de patrones fotométricos. Por tanto, utiliza dichos patrones fotométricos
para ayudarse en la tarea de detectar esquinas.

\nonumsection{Ejercicio 11}

\question{¿Sería adecuado usar como descriptor de un punto Harris los
valores de los píxeles de su región de soporte? Identifique
ventajas, inconvenientes y mecanismos de superación de estos
últimos.}

\answer

\nonumsection{Ejercicio 12}

\question{Describa un par de criterios que sirvan para seleccionar
parejas de puntos en correspondencias (``matching'') a partir de
descriptores de regiones extraídos de dos imágenes. ¿Por qué no es
posible garantizar que todas las parejas son correctas?}

\answer

Existen algunos criterios para hacer \textit{matching} entre puntos:

\begin{itemize}[label=\textbullet]
	\item \textbf{Fuerza bruta con \textit{Cross Check}}: Este criterio es
	el más intuitivo. Simplemente consiste en unir cada descriptor $d_{1i}$ de una imagen
	con cada descriptor $d_{2j}$ de la otra iamgen, y escoger solo aquellas parejas
	de correspondencias en las que el punto más cercano a $d_{1i}$ sea $d_{2j}$ y viceversa
	(se tiene que dar en ambos sentidos; si solo se da en uno, se descarta).
	\item \textbf{Lowe-Average-2NN }: Para cada descriptor de una imagen, se escogen
	los dos descriptores más cercanos de la otra imagen. Se mide la distancia al
	mejor $dist_1$ y al segundo mejor $dist_2$, y solo se acepta el \textit{match} con
	el mejor si $dist_1 < 0.8dist_2$ (según el \textit{paper} original). En caso contrario,
	se descarta el \textit{match}.
\end{itemize}

Estos criterios sin embargo no garantizan que todas las parejas sean
correctas. Por ejemplo, puede darse el caso de que no exista un mejor
punto para uno dado, esta correspondencia sea bastante mala. También podría darse
que en la aparezcan elementos simétricos o con patrones repetidos, lo cuál
puede llegar a complicar el proceso del \textit{matching} al haber más de
una posible correspondencia.

\nonumsection{Ejercicio 13}

\question{¿Cuál es el objetivo principal del uso de la técnica RANSAC en
el cálculo de una homografía? Justificar la respuesta.}

\answer

El objetivo de RANSAC es estimar mejor las homografías. Para ello, solo tiene
en cuenta los \textit{matches} que considere como \textit{inliers}, descartando
por tanto los \textit{outliers}. Los \textit{inliers} son las correspondencias
que son consistentes con la homografía estimada, mientras que los \textit{outliers}
son aquellas correspondencias que no lo son. RANSAC intenta estimar la homografía
con el menor número de \textit{outliers}, de forma que el impacto que tengan los
malos \textit{matches} sea pequeño en la homografía final. Así, las homografías
estimadas son más resistentes a correspondencias inciertas o debidas al ruido.

\nonumsection{Ejercicio 14}

\question{Si tengo 4 imágenes de una escena de manera que se solapan la
1-2, 2-3 y 3-4. ¿Cuál es el número mínimo de parejas de puntos en
correspondencias necesarios para montar un mosaico? Justificar la
respuesta.}

\answer

\nonumsection{Ejercicio 15}

\question{¿En la confección de un mosaico con proyección rectangular es
esperable que aparezcan deformaciones geométricas de la escena
real? ¿Cuáles y por qué? ¿Bajo qué condiciones esas deformaciones
podrían no estar presentes? Justificar la respuesta.}

\answer

\newpage

\bibliographystyle{plain}
\nocite{*}
\bibliography{mybib}

\end{document}

