\documentclass[11pt,a4paper]{article}
\usepackage[spanish,es-nodecimaldot]{babel}	% Utilizar español
\usepackage[utf8]{inputenc}					% Caracteres UTF-8
\usepackage{graphicx}						% Imagenes
\usepackage[hidelinks]{hyperref}			% Poner enlaces sin marcarlos en rojo
\usepackage{fancyhdr}						% Modificar encabezados y pies de pagina
\usepackage{float}							% Insertar figuras
\usepackage[textwidth=390pt]{geometry}		% Anchura de la pagina
\usepackage[nottoc]{tocbibind}				% Referencias (no incluir num pagina indice en Indice)
\usepackage{enumitem}						% Permitir enumerate con distintos simbolos
\usepackage[T1]{fontenc}					% Usar textsc en sections
\usepackage{amsmath}						% Símbolos matemáticos

% Comando para poner el nombre de la asignatura
\newcommand{\asignatura}{Visión por Computador}
\newcommand{\autor}{Vladislav Nikolov Vasilev}
\newcommand{\titulo}{Trabajo 3}
\newcommand{\subtitulo}{Cuestiones de teoría}

\newcommand{\answer}{\noindent\textbf{Solución}}
\newcommand{\question}[1]{\noindent\textbf{#1}}
\newcommand{\nonumsection}[1]{\section*{#1}\addcontentsline{toc}{section}{#1}}

% Configuracion de encabezados y pies de pagina
\pagestyle{fancy}
\lhead{\autor{}}
\rhead{\asignatura{}}
\lfoot{Grado en Ingeniería Informática}
\cfoot{}
\rfoot{\thepage}
\renewcommand{\headrulewidth}{0.4pt}		% Linea cabeza de pagina
\renewcommand{\footrulewidth}{0.4pt}		% Linea pie de pagina

\begin{document}
\pagenumbering{gobble}

% Pagina de titulo
\begin{titlepage}

\begin{minipage}{\textwidth}

\centering

\includegraphics[scale=0.5]{img/ugr.png}\\

\textsc{\Large \asignatura{}\\[0.2cm]}
\textsc{GRADO EN INGENIERÍA INFORMÁTICA}\\[1cm]

\noindent\rule[-1ex]{\textwidth}{1pt}\\[1.5ex]
\textsc{{\Huge \titulo\\[0.5ex]}}
\textsc{{\Large \subtitulo\\}}
\noindent\rule[-1ex]{\textwidth}{2pt}\\[3.5ex]

\end{minipage}

\vspace{0.5cm}

\begin{minipage}{\textwidth}

\centering

\textbf{Autor}\\ {\autor{}}\\[2.5ex]
\textbf{Rama}\\ {Computación y Sistemas Inteligentes}\\[2.5ex]
\vspace{0.3cm}

\includegraphics[scale=0.3]{img/etsiit.jpeg}

\vspace{0.7cm}
\textsc{Escuela Técnica Superior de Ingenierías Informática y de Telecomunicación}\\
\vspace{1cm}
\textsc{Curso 2019-2020}
\end{minipage}
\end{titlepage}

\pagenumbering{arabic}
\tableofcontents
\thispagestyle{empty}				% No usar estilo en la pagina de indice

\newpage

\setlength{\parskip}{1em}

\nonumsection{Ejercicio 1}

\question{¿Cuál es la transformación más fuerte de la geometría de una
escena que puede introducirse al tomar una foto de ella? Dar algún
ejemplo.}

\answer

La transformación más fuerte de la geometría de una escena que puede introducirse
al tomar una foto es la \textbf{transformación proyectiva}, también conocida como
\textbf{transformación de perspectiva}. Esta transformación es capaz de conservar
las líneas rectas, pero no conserva ni los ángulos, ni los tamaños de los objetos
de la escena ni las líneas paralelas, ya que estos elementos dependen de la perspectiva
del observador dentro de la escena. Dos observadores situados en la misma escena
pero en distintos lugares no tomarán la misma foto si observan al mismo punto.

Un ejemplo muy claro de esto es la típica foto de las vías del tren. En la realidad,
las vías forman líneas rectas, pero al tomar una fotografía de ellas desde el suelo,
las líneas no son paralelas; es más, parece que se cortan en el infinito (cosa que
en la realidad no sucede). Otro ejemplo sería por ejemplo una foto tomada en un pasillo.
Las líneas que separan el suelo de las paredes de cada lado del pasillo no se juntan
en la realidad en ningún momento, pero por la perspectiva con la que se ha tomado la foto,
tal y como pasó en el caso de las vías, podría parecer que las líneas se juntasen en el
infinito.


\nonumsection{Ejercicio 2}

\question{¿Por qué es necesario usar el plano proyectivo para estudiar las
transformaciones en las imágenes de fotos de escenas? Dar algún
ejemplo.}

\answer

\nonumsection{Ejercicio 3}

\question{Sabemos que en el plano proyectivo un punto no existe en el
sentido del plano afín, sino que se define por una clase de
equivalencia de vectores definida por $\lbrace k(x,y,1),k \neq 0 \rbrace$. Razone usando
las coordenadas proyectivas de los puntos afines de una recta que
pase por el $(0,0)$ del plano afín y verifique que los punto de la
recta del infinito del plano proyectivo son necsariamente vectores
del tipo $(*,*,0)$ con $*=$cualquier número.}

\answer

\nonumsection{Ejercicio 4}

\question{¿Qué propiedades de la geometría de un plano quedan invariantes
cuando se toma una foto de él? Justificar la respuesta.}

\answer

\nonumsection{Ejercicio 5}



\nonumsection{Ejercicio 6}

\question{¿Cuál es el mínimo número de escalares necesarios para fijar
una homografía general? ¿Y si la homografía es afín? Justificar la
respuesta.}

\answer

\nonumsection{Ejercicio 7}

\question{Defina una homografía entre planos proyectivos que haga que el
punto $(3,0,2)$ del plano proyectivo-1 se transforme en un punto de
la recta del infinito del plano proyectivo-2? Justificar la
respuesta.}

\answer

\nonumsection{Ejercicio 8}



\nonumsection{Ejercicio 9}

\question{¿Cuáles son las propiedades necesarias y suficientes para que
una matriz defina un movimiento geométrico no degenerado entre
planos? Justificar la respuesta.}

\answer

\nonumsection{Ejercicio 10}

\question{¿Qué información de la imagen usa el detector de Harris para
seleccionar puntos? ¿El detector de Harris detecta patrones
geométricos o fotométricos? Justificar la contestación.}

\answer

\nonumsection{Ejercicio 11}

\question{¿Sería adecuado usar como descriptor de un punto Harris los
valores de los píxeles de su región de soporte? Identifique
ventajas, inconvenientes y mecanismos de superación de estos
últimos.}

\answer

\nonumsection{Ejercicio 12}

\question{Describa un par de criterios que sirvan para seleccionar
parejas de puntos en correspondencias (``matching'') a partir de
descriptores de regiones extraídos de dos imágenes. ¿Por qué no es
posible garantizar que todas las parejas son correctas?}

\answer

\nonumsection{Ejercicio 13}

\question{¿Cuál es el objetivo principal del uso de la técnica RANSAC en
el cálculo de una homografía? Justificar la respuesta.}

\answer

\nonumsection{Ejercicio 14}

\question{Si tengo 4 imágenes de una escena de manera que se solapan la
1-2, 2-3 y 3-4. ¿Cuál es el número mínimo de parejas de puntos en
correspondencias necesarios para montar un mosaico? Justificar la
respuesta.}

\answer

\nonumsection{Ejercicio 15}

\question{¿En la confección de un mosaico con proyección rectangular es
esperable que aparezcan deformaciones geométricas de la escena
real? ¿Cuáles y por qué? ¿Bajo qué condiciones esas deformaciones
podrían no estar presentes? Justificar la respuesta.}

\answer

\newpage

\begin{thebibliography}{5}

\end{thebibliography}

\end{document}

