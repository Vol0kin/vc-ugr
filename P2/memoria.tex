\documentclass[11pt,a4paper]{article}
\usepackage[spanish,es-nodecimaldot]{babel}	% Utilizar español
\usepackage[utf8]{inputenc}					% Caracteres UTF-8
\usepackage{graphicx}						% Imagenes
\usepackage[hidelinks]{hyperref}			% Poner enlaces sin marcarlos en rojo
\usepackage{fancyhdr}						% Modificar encabezados y pies de pagina
\usepackage{float}							% Insertar figuras
\usepackage[textwidth=390pt]{geometry}		% Anchura de la pagina
\usepackage[nottoc]{tocbibind}				% Referencias (no incluir num pagina indice en Indice)
\usepackage{enumitem}						% Permitir enumerate con distintos simbolos
\usepackage[T1]{fontenc}					% Usar textsc en sections
\usepackage{amsmath}						% Símbolos matemáticos

% Comando para poner el nombre de la asignatura
\newcommand{\asignatura}{Visión por Computador}
\newcommand{\autor}{Vladislav Nikolov Vasilev}
\newcommand{\titulo}{Práctica 2}
\newcommand{\subtitulo}{Redes Neuronales Convolucionales}

\usepackage{listings}
\usepackage{xcolor}
 
\definecolor{codegreen}{rgb}{0,0.6,0}
\definecolor{codegray}{rgb}{0.5,0.5,0.5}
\definecolor{codepurple}{rgb}{0.58,0,0.82}
\definecolor{backcolour}{rgb}{0.95,0.95,0.92}
 
\lstdefinestyle{mystyle}{
    backgroundcolor=\color{backcolour},   
    commentstyle=\color{codegreen},
    keywordstyle=\color{magenta},
    numberstyle=\tiny\color{codegray},
    stringstyle=\color{codepurple},
    basicstyle=\ttfamily\footnotesize,
    breakatwhitespace=false,         
    breaklines=true,                 
    captionpos=b,                    
    keepspaces=true,                 
    numbers=left,                    
    numbersep=5pt,                  
    showspaces=false,                
    showstringspaces=false,
    showtabs=false,                  
    tabsize=4,
    language=Python,
    literate={ñ}{{\~n}}1
}

\lstset{style=mystyle}

% Configuracion de encabezados y pies de pagina
\pagestyle{fancy}
\lhead{\autor{}}
\rhead{\asignatura{}}
\lfoot{Grado en Ingeniería Informática}
\cfoot{}
\rfoot{\thepage}
\renewcommand{\headrulewidth}{0.4pt}		% Linea cabeza de pagina
\renewcommand{\footrulewidth}{0.4pt}		% Linea pie de pagina

\begin{document}
\pagenumbering{gobble}

% Pagina de titulo
\begin{titlepage}

\begin{minipage}{\textwidth}

\centering

\includegraphics[scale=0.5]{img/ugr.png}\\

\textsc{\Large \asignatura{}\\[0.2cm]}
\textsc{GRADO EN INGENIERÍA INFORMÁTICA}\\[1cm]

\noindent\rule[-1ex]{\textwidth}{1pt}\\[1.5ex]
\textsc{{\Huge \titulo\\[0.5ex]}}
\textsc{{\Large \subtitulo\\}}
\noindent\rule[-1ex]{\textwidth}{2pt}\\[3.5ex]

\end{minipage}

\vspace{0.5cm}

\begin{minipage}{\textwidth}

\centering

\textbf{Autor}\\ {\autor{}}\\[2.5ex]
\textbf{Rama}\\ {Computación y Sistemas Inteligentes}\\[2.5ex]
\vspace{0.3cm}

\includegraphics[scale=0.3]{img/etsiit.jpeg}

\vspace{0.7cm}
\textsc{Escuela Técnica Superior de Ingenierías Informática y de Telecomunicación}\\
\vspace{1cm}
\textsc{Curso 2019-2020}
\end{minipage}
\end{titlepage}

\pagenumbering{arabic}
\tableofcontents
\thispagestyle{empty}				% No usar estilo en la pagina de indice

\newpage

\setlength{\parskip}{1em}

\section{\textsc{BaseNet en CIFAR100}}

Antes de empezar con la traducción de la arquitectura proporcionada de BaseNet, hace falta establecer la forma
de la entrada de la primera capa de la red. Esto es necesario, ya que el modelo necesita conocer dicho tamaño para poder
ser compilado sin ningún tipo de error. Como las imágenes de \textit{CIFAR100} tienen un tamaño de $32 \times 32$
píxels, y tienen 3 canales, la dimensión de la entrada va a ser la siguiente:

\begin{lstlisting}
# Tamaño de la entrada
input_shape = (32, 32, 3)
\end{lstlisting}

Una vez definida la forma de la entrada, ya se puede empezar a hacer la traducción a código. El resultado es el siguiente:

\begin{lstlisting}
# Creacion del modelo
model = Sequential()
model.add(Conv2D(6, kernel_size=(5, 5), padding='valid',
				 input_shape=input_shape))
model.add(Activation('relu'))
model.add(MaxPooling2D(pool_size=(2, 2)))

model.add(Conv2D(16, kernel_size=(5, 5), padding='valid'))
model.add(Activation('relu'))
model.add(MaxPooling2D(pool_size=(2, 2)))

model.add(Flatten())
model.add(Dense(units=50))
model.add(Activation('relu'))
model.add(Dense(units=25))
model.add(Activation('softmax'))
\end{lstlisting}

BaseNet es un modelo secuencial, así que empezamos indicando eso. A continuación, añadimos el primer módulo convolucional.
Este se compone de una convolución 2D con un \textit{kernel} de $5 \times 5$, una función de activación no lineal (RELU en
este caso) y un MaxPooling de tamaño $2 \times 2$. El parámetro $padding = valid$ de $Conv2D$ indica que solo se tiene que
aplicar la convolución allá donde se pueda ajustar el \textit{kernel}; es decir, como en las regiones de los bordes no se
puede, se van a ignorar estas zonas, lo cuál implica que la salida no va a tener el mismo tamaño que la entrada. Este módulo
convolucional se repite otra vez. Después de eso, nos encontramos con las capas densas, las cuáles van a actuar como
clasificador. La capa $Flatten$ es necesario ponerla, ya que coge la salida de la anterior y la aplana, dejándola como
un vector de características que sirve como entrada al modelo denso. La última capa, la de activación \textit{softmax}
es la que va a dar la salida, clasificando las entradas.


\section{\textsc{Mejora del modelo}}

\section{\textsc{Transferencia de modelos y ajuste fino con ResNet50 para la base de datos Caltech-UCSD}}

\newpage

\begin{thebibliography}{5}

\bibitem{nombre-referencia}
Texto referencia
\\\url{https://url.referencia.com}

\end{thebibliography}

\end{document}

