\documentclass[11pt,a4paper]{article}
\usepackage[spanish,es-nodecimaldot]{babel}	% Utilizar español
\usepackage[utf8]{inputenc}					% Caracteres UTF-8
\usepackage{graphicx}						% Imagenes
\usepackage[hidelinks]{hyperref}			% Poner enlaces sin marcarlos en rojo
\usepackage{fancyhdr}						% Modificar encabezados y pies de pagina
\usepackage{float}							% Insertar figuras
\usepackage[textwidth=390pt]{geometry}		% Anchura de la pagina
\usepackage[nottoc]{tocbibind}				% Referencias (no incluir num pagina indice en Indice)
\usepackage{enumitem}						% Permitir enumerate con distintos simbolos
\usepackage[T1]{fontenc}					% Usar textsc en sections
\usepackage{amsmath}						% Símbolos matemáticos

% Comando para poner el nombre de la asignatura
\newcommand{\asignatura}{Visión por Computador}
\newcommand{\autor}{Vladislav Nikolov Vasilev}
\newcommand{\titulo}{Trabajo 1}
\newcommand{\subtitulo}{Filtrado y Detección de Regiones}

% Configuracion de encabezados y pies de pagina
\pagestyle{fancy}
\lhead{\autor{}}
\rhead{\asignatura{}}
\lfoot{Grado en Ingeniería Informática}
\cfoot{}
\rfoot{\thepage}
\renewcommand{\headrulewidth}{0.4pt}		% Linea cabeza de pagina
\renewcommand{\footrulewidth}{0.4pt}		% Linea pie de pagina

\begin{document}
\pagenumbering{gobble}

% Pagina de titulo
\begin{titlepage}

\begin{minipage}{\textwidth}

\centering

\includegraphics[scale=0.5]{img/ugr.png}\\

\textsc{\Large \asignatura{}\\[0.2cm]}
\textsc{GRADO EN INGENIERÍA INFORMÁTICA}\\[1cm]

\noindent\rule[-1ex]{\textwidth}{1pt}\\[1.5ex]
\textsc{{\Huge \titulo\\[0.5ex]}}
\textsc{{\Large \subtitulo\\}}
\noindent\rule[-1ex]{\textwidth}{2pt}\\[3.5ex]

\end{minipage}

\vspace{0.5cm}

\begin{minipage}{\textwidth}

\centering

\textbf{Autor}\\ {\autor{}}\\[2.5ex]
\textbf{Rama}\\ {Computación y Sistemas Inteligentes}\\[2.5ex]
\vspace{0.3cm}

\includegraphics[scale=0.3]{img/etsiit.jpeg}

\vspace{0.7cm}
\textsc{Escuela Técnica Superior de Ingenierías Informática y de Telecomunicación}\\
\vspace{1cm}
\textsc{Curso 2019-2020}
\end{minipage}
\end{titlepage}

\pagenumbering{arabic}
\tableofcontents
\thispagestyle{empty}				% No usar estilo en la pagina de indice

\newpage

\setlength{\parskip}{1em}

\section{\textsc{Ejercicio sobre filtros básicos}}

\noindent \textbf{USANDO LAS FUNCIONES DE OPENCV}: escribir funciones que implementen los siguientes puntos:

\begin{enumerate}[label=\textbf{\Alph*)}]
	\item \textbf{El cálculo de la convolución de una imagen con una máscara 2D. Usar una Gaussiana 2D (GaussianBlur)
	y máscaras 1D dadas por getDerivKernels). Mostrar ejemplos con distintos tamaños de máscara, valores de
	sigma y condiciones de contorno. Valorar los resultados.}
	\item \textbf{Usar la función Laplacian para el cálculo de la convolución 2D con una máscara normalizada de
	Laplaciana-de-Gaussiana de tamaño variable. Mostrar ejemplos de funcionamiento usando dos tipos de
	bordes y dos valores de sigma: 1 y 3.}
\end{enumerate}

\subsection{Apartado A}

\subsection{Apartado B}

\newpage

\section{\textsc{Ejercicio sobre pirámides y detección de regiones}}

\noindent \textbf{IMPLEMENTAR} funciones para las siguiente tareas:

\begin{enumerate}[label=\textbf{\Alph*)}]
	\item \textbf{Una función que genere una representación en pirámide Gaussiana de 4 niveles de una
	imagen. Mostrar ejemplos de funcionamiento usando bordes y justificar la elección de los parámetros.}
	\item \textbf{Una función que genere una representación en pirámide Laplaciana de 4 niveles de una imagen.
	Mostrar ejemplos de funcionamiento usando bordes.}
	\item \textbf{Construir un espacio de escalas Laplaciano para implementar la búsqueda de regiones usando el siguiente
	algoritmo:}
	\begin{enumerate}[label=\textbf{\alph*.}]
		\item Fijar sigma
		\item Repetir para N escalas
		\item Realizar supresión de no-máximos en cada escala
		\begin{enumerate}[label=\roman*.]
			\item Filtrar la imagen con la Laplaciana-Gaussiana normalizada en escala
			\item Guardar el cuadrado de la respuesta para el actual nivel del espacio de escalas
			\item Incrementar el valor de sigma por un coeficiente k.( 1.2-1.4)
		\end{enumerate}
		\item Mostrar las regiones encontradas en sus correspondientes escalas. Dibujar círculos con radio proporcional a
		la escala.
	\end{enumerate}
\end{enumerate}

\subsection{Apartado A}

\subsection{Apartado B}

\subsection{Apartado C}

\newpage

\section{\textsc{Imágenes híbridas}}

\noindent \textbf{Mezclando adecuadamente una parte de las frecuencias altas de una imagen con una parte de
las frecuencias bajas de otra imagen, obtenemos una imagen híbrida que admite distintas interpretaciones a distintas
distancias (ver hybrid images project page).}

\noindent \textbf{Para seleccionar la parte de frecuencias altas y bajas que nos quedamos
de cada una de las imágenes usaremos el parámetro sigma del núcleo/máscara de alisamiento gaussiano que usaremos.
A mayor valor de sigma mayor eliminación de altas frecuencias en la imagen convolucionada. Para una buena
implementación elegir dicho valor de forma separada para cada una de las dos imágenes (ver las recomendaciones
dadas en el paper de Oliva et al.). Recordar que las máscaras 1D siempre deben tener de longitud un número impar.}

\noindent \textbf{Implementar una función que genere las imágenes de baja y alta frecuencia a partir de las
parejas de imágenes (solo en la versión de imágenes de gris). El valor de sigma más adecuado para cada pareja
habrá que encontrarlo por experimentación.}

\begin{enumerate}
	\item Escribir una función que muestre las tres imágenes (alta, baja e híbrida) en una misma ventana.
	(Recordar que las imágenes después de una convolución contienen número flotantes que pueden ser positivos y negativos)
	\item Realizar la composición con al menos 3 de las parejas de imágenes
	\item Construir pirámides gaussianas de al menos 4 níveles con las imágenes resultado. Explicar el efecto que se observa.
\end{enumerate}

\subsection{Apartado 1}

\subsection{Apartado 2}

\subsection{Apartado 3}

\newpage

\section{Bonus}

\subsection{Convolución 2D propia}

\noindent \textbf{Implementar con código propio la convolución 2D con cualquier máscara 2D de números reales usando
máscaras separables.}

\subsection{Imágenes híbridas a color}

\noindent \textbf{Realizar todas las parejas de imágenes híbridas en su formato a color (solo se tendrá en cuenta
si la versión de gris es correcta).}

\subsection{Imágen híbrida propia}

\noindent \textbf{Realizar una imagen híbrida con al menos una pareja de imágenes de su elección que hayan
sido extraídas de imágenes más grandes. Justifique la elección y todos los pasos que realiza.}

\newpage

\begin{thebibliography}{5}

\bibitem{nombre-referencia}
Texto referencia
\\\url{https://url.referencia.com}

\end{thebibliography}

\end{document}

