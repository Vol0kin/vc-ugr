\documentclass[11pt,a4paper]{article}
\usepackage[spanish,es-nodecimaldot]{babel}	% Utilizar español
\usepackage[utf8]{inputenc}					% Caracteres UTF-8
\usepackage{graphicx}						% Imagenes
\usepackage[hidelinks]{hyperref}			% Poner enlaces sin marcarlos en rojo
\usepackage{fancyhdr}						% Modificar encabezados y pies de pagina
\usepackage{float}							% Insertar figuras
\usepackage[textwidth=390pt]{geometry}		% Anchura de la pagina
\usepackage[nottoc]{tocbibind}				% Referencias (no incluir num pagina indice en Indice)
\usepackage{enumitem}						% Permitir enumerate con distintos simbolos
\usepackage[T1]{fontenc}					% Usar textsc en sections
\usepackage{amsmath}						% Símbolos matemáticos

% Comando para poner el nombre de la asignatura
\newcommand{\asignatura}{Visión por Computador}
\newcommand{\autor}{Vladislav Nikolov Vasilev}
\newcommand{\titulo}{Trabajo 1}
\newcommand{\subtitulo}{Cuestiones de teoría}
\newcommand{\answer}{\noindent\textbf{Solución}}
\newcommand{\question}[1]{\noindent\textbf{#1}}
\newcommand{\nonumbersection}[1]{\section*{#1}\addcontentsline{toc}{section}{#1}}

% Configuracion de encabezados y pies de pagina
\pagestyle{fancy}
\lhead{\autor{}}
\rhead{\asignatura{}}
\lfoot{Grado en Ingeniería Informática}
\cfoot{}
\rfoot{\thepage}
\renewcommand{\headrulewidth}{0.4pt}		% Linea cabeza de pagina
\renewcommand{\footrulewidth}{0.4pt}		% Linea pie de pagina

\begin{document}
\pagenumbering{gobble}

% Pagina de titulo
\begin{titlepage}

\begin{minipage}{\textwidth}

\centering

\includegraphics[scale=0.5]{img/ugr.png}\\

\textsc{\Large \asignatura{}\\[0.2cm]}
\textsc{GRADO EN INGENIERÍA INFORMÁTICA}\\[1cm]

\noindent\rule[-1ex]{\textwidth}{1pt}\\[1.5ex]
\textsc{{\Huge \titulo\\[0.5ex]}}
\textsc{{\Large \subtitulo\\}}
\noindent\rule[-1ex]{\textwidth}{2pt}\\[3.5ex]

\end{minipage}

\vspace{0.5cm}

\begin{minipage}{\textwidth}

\centering

\textbf{Autor}\\ {\autor{}}\\[2.5ex]
\textbf{Rama}\\ {Computación y Sistemas Inteligentes}\\[2.5ex]
\vspace{0.3cm}

\includegraphics[scale=0.3]{img/etsiit.jpeg}

\vspace{0.7cm}
\textsc{Escuela Técnica Superior de Ingenierías Informática y de Telecomunicación}\\
\vspace{1cm}
\textsc{Curso 2018-2019}
\end{minipage}
\end{titlepage}

\pagenumbering{arabic}
\tableofcontents
\thispagestyle{empty}				% No usar estilo en la pagina de indice

\newpage

\setlength{\parskip}{1em}

\nonumbersection{Ejercicio 1}

\question{Diga en una sola frase cuál cree que es el objetivo principal de la Visión por Computador. Diga también cuál es la
principal propiedad de las imágenes de cara a la creación algoritmos que la procesen.}

\answer

\nonumbersection{Ejercicio 2}

\question{Expresar las diferencias y semejanzas entre las operaciones de correlación y convolución. Dar una interpretación de
cada una de ellas que en el contexto de uso en visión por computador.}

\answer

\nonumbersection{Ejercicio 3}

\question{¿Cuál es la diferencia ``esencial'' entre el filtro de convolución y el de mediana? Justificar la respuesta.}

\answer

\nonumbersection{Ejercicio 4}

\question{Identifique el ``mecanismo concreto'' que usa un filtro de máscara para transformar una imagen.}

\answer

\nonumbersection{Ejercicio 5}

\question{¿De qué depende que una máscara de convolución pueda ser implementada por convoluciones 1D? Justificar la respuesta.}

\answer

\nonumbersection{Ejercicio 6}

\question{Identificar las diferencias y consecuencias desde el punto de vista teórico y de la implementación entre:
\begin{enumerate}[label=\alph*)]
	\item Primero alisar la imagen y después calcular las derivadas sobre la
imagen alisada.
	\item Primero calcular las imágenes derivadas y después alisar dichas
imágenes.
\end{enumerate}
Justificar los argumentos.
}

\answer

\nonumbersection{Ejercicio 7}

\question{Identifique las funciones de las que podemos extraer pesos correctos
para implementar de forma eficiente la primera derivada de una imagen.
Suponer alisamiento Gaussiano.}

\answer

\nonumbersection{Ejercicio 8}

\question{Identifique las funciones de las que podemos extraer pesos correctos
para implementar de forma eficiente la Laplaciana de una imagen. Suponer
alisamiento Gaussiano.}

\answer

\nonumbersection{Ejercicio 9}

\question{Suponga que le piden implementar de forma eficiente un algoritmo para
el cálculo de la derivada de primer orden sobre una imagen usando
alisamiento Gaussiano. Enumere y explique los pasos necesarios para
llevarlo a cabo.}

\answer

\nonumbersection{Ejercicio 10}

\question{Identifique semejanzas y diferencias entre la pirámide gaussiana y
el espacio de escalas de una imagen, ¿cuándo usar una u otra? Justificar
los argumentos.}

\answer

\nonumbersection{Ejercicio 11}

\question{¿Bajo qué condiciones podemos garantizar una perfecta reconstrucción
de una imagen a partir de su pirámide Laplaciana? Dar argumentos y
discutir las opciones que considere necesario.}

\answer

\nonumbersection{Ejercicio 12}

\question{ ¿Cuáles son las contribuciones más relevantes del algoritmo de
Canny al cálculo de los contornos sobre una imagen? ¿Existe alguna
conexión entre las máscaras de Sobel y el algoritmo de Canny? Justificar
la respuesta.}

\answer

\nonumbersection{Ejercicio 13}

\question{- Identificar pros y contras de k-medias como mecanismo para crear un
vocabulario visual a partir del cual poder caracterizar patrones. ¿Qué
ganamos y que perdemos? Justificar los argumentos.}

\answer

\nonumbersection{Ejercicio 14}

\question{Identifique pros y contras del modelo de “Bolsa de Palabras” como
mecanismo para caracterizar el contenido de una imagen. ¿Qué ganamos y
que perdemos? Justificar los argumentos.}

\answer

\nonumbersection{Ejercicio 15}

\question{Suponga que dispone de un conjunto de imágenes de dos tipos de
clases bien diferenciadas. Suponga que conoce como implementar de forma
eficiente el cálculo de las derivadas hasta el orden N de la imagen.
Describa como crear un algoritmo que permita diferenciar, con garantías,
imágenes de ambas clases. Justificar cada uno de los pasos que proponga.}

\answer

\newpage

\begin{thebibliography}{5}

\bibitem{nombre-referencia}
Texto referencia
\\\url{https://url.referencia.com}

\end{thebibliography}

\end{document}

